

% The standard graphics inclusion package
\usepackage{graphicx}
% Set up how figure and table captions are displayed
%\usepackage{caption}
%\captionsetup{%
%  font=footnotesize,% set font size to footnotesize
%  labelfont=bf % bold label (e.g., Figure 3.2) font
%}
\graphicspath{{./imgs/}}


\addtokomafont{caption}{\small}
\addtokomafont{captionlabel}{\bfseries}


\usepackage{xcolor}
\definecolor{colortitulos}{RGB}{0,20,75}
\addtokomafont{sectioning}{\color{colortitulos}}

%csquotes
% uso: \enquote{text}
% ¿Por qué strict?
\usepackage[strict=true,autostyle=true]{csquotes}


% Add todo notes in the margin of the document
\usepackage[
%  disable, %turn off todonotes
  colorinlistoftodos, %enable a coloured square in the list of todos
  textwidth=\marginparwidth, %set the width of the todonotes
%  textsize=scriptsize, %size of the text in the todonotes
  textsize=tiny,
  obeyFinal, % 
  spanish,
  ]{todonotes}
  

% Make the standard latex tables look so much better
% An extended implementation of the array and tabular environments which extends the options for column formats, and provides "programmable" format specifications.
% The pack­age en­hances the qual­ity of ta­bles in LaTeX, pro­vid­ing ex­tra com­mands as well as be­hind-the-scenes op­ti­mi­sa­tion. Guide­lines are given as to what con­sti­tutes a good ta­ble in this con­text. 
\usepackage{array,booktabs}


\usepackage[allowlitunits]{siunitx}
%\selectlanguage{spanish} % También está como opción global de clase de documento..
\sisetup{group-digits = true}
\sisetup{group-minimum-digits = 5} % hasta 4 no separa de a 3.
\sisetup{group-separator = {\,}} % Un espacio pequeño
\sisetup{output-decimal-marker = {,}} % La coma para separar enteros de decimales
\DeclareSIUnit\pixel{px}







\usepackage{microtype}

\usepackage{bm} %bold symbols in math mode

\usepackage{multirow}

\usepackage{tabulary}
%\usepackage{tabularx}

\usepackage{pdflscape}
\usepackage{afterpage}

\KOMAoptions{toc=bibliography}

\usepackage{subcaption}

 \interfootnotelinepenalty=10000 %% Completely prevent breaking of footnotes
 
 
   \usepackage{url} % Ver si es prescindible; tal vez con hyperref basta-
 
 \usepackage[backend=biber,style=numeric,sorting=none,backref=true,hyperref=true]{biblatex}
\bibliography{tfinal.bib} %PUEDE HABER MÁS DE UNO?
 \renewcommand{\finentrypunct}{} % para que no ponga un punto al final de cada entrada de la bibliografía. Tal vez no importa si hay hipervínculos a las urls.
 \DefineBibliographyStrings{spanish}{%
   backrefpage = {Citado en página},% originally "cited on page" vid.
   backrefpages = {Citado en páginas},% originally "cited on pages"
 }
  
  

%  
%\usepackage[hidelinks]{hyperref}
%\hypersetup{%
% pdfpagelabels=true,%
% plainpages=false,%
% pdfauthor={Pablo Daniel AGUADO},%
% pdftitle={Una estrategia para la clasificación óptica de almendras},%
% pdfsubject={Trabajo Final, UNSJ},%
% bookmarksnumbered=true,%
% colorlinks=false,%
% citecolor=black,%
% filecolor=black,%
% linkcolor=black,% you should probably change this to black before printing
% urlcolor=black,%
% pdfstartview=FitH%
%}
  
\usepackage[hidelinks]{hyperref}
\hypersetup{%
  pdfpagelabels=true,%
  plainpages=false,%
  pdfauthor={Pablo Daniel AGUADO},%
  pdftitle={Una estrategia para la clasificación óptica de almendras},%
  pdfsubject={Trabajo Final, UNSJ},%
  bookmarksnumbered=true,%
  pdfstartview=FitH%
}%falta índice en marcadores.
 
 % Para otra vez: Usar \autoref{label} de hyperref, en lugar de \ref{}