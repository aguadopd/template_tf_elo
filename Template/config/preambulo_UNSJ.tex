% Clase del documento. Koma-Script Report de base.
\documentclass[draft,spanish,twoside=no]{scrreprt}
%\documentclass[final,spanish,twoside=no]{scrreprt}
\KOMAoptions{numbers=noendperiod} %NO ESTA FUNCIONANDO
%\KOMAoptions{headinclude=true}
%\KOMAoptions{footinclude=true}
%\KOMAoptions{DIV=15}



% IDIOMA
% Polyglossia - An alternative to babel for XeLaTeX and LuaLaTeX
% https://ctan.org/pkg/polyglossia
% Este paquete es para mejor soporte de idiomas.
%\usepackage[]{polyglossia}
%\setdefaultlanguage[]{spanish}
%\setotherlanguage[variant=american]{english}
% Para escribir en inglés, usar el comando \textenglish{texto}
% o el entorno english: \begin{english}[opciones]{}\end{english}

\usepackage[utf8]{inputenc}
% Make latex understand and use the typographic
% rules of the language used in the document.
\usepackage[main=spanish, english]{babel}
\babeltags{en = english} % Y entonces \texten{texto} servirá para inglés. igual usaremos macro \ingles{texto}.
%Para cosas largas: \begin{en} y \end{en}

%\addto\captionsspanish{
%	\def\tablename{Tabla}
%	\def\listtablename{\'Indice de tablas}
%}


% Para usar 'tablas' en lugar de 'cuadros'.
% Alternativa si usamos Koma-Script:
\renewcaptionname{spanish}{\listtablename}{Índice de tablas}
\renewcaptionname{spanish}{\tablename}{Tabla}
% Alternativa propuesta por Polyglossia:
%\usepackage{etoolbox}
%\gappto\captionsspanish{\renewcommand{\tablename}{Tabla}}
%\gappto\captionsspanish{\renewcommand{\listtablename}{Índice de tablas}}



%Tamaño de papel: A4 

%Márgenes: Superior = Inferior = 2,5cm; Izquierdo = 2cm; Derecho = 2cm 
\usepackage[left=2cm,top=2.5cm,right=2cm,bottom=2.5cm,bindingoffset=0cm,footskip=1.5cm]{geometry}

%Impresión en doble faz.
% Está arriba con twoside=no

%Letra tamaño 12. Tipo Arial o Times New Roman. 
%\usepackage{helvet}
%\usepackage[scaled]{helvet} 
%Helvetica is actually somewhat larger than other typefaces of the same nominal size.  As a result, mixing, e.g., Times and Helvetica within running text may look bad. This  can  be  fixed  by  loading  the  package  with  the  option [scaled=〈scale〉], for instance: \usepackage[scaled=.92]{helvet}.  As a result, the font family phv (Helvetica) will be scaled down to 92% of its ‘natural’ size, which is suitable for use with Adobe Times.  Specifying [scaled] alone is equivalent to [scaled=0.95].

\usepackage{newtxtext, newtxmath} % JUSTIFICAR
\usepackage[zerostyle=d,scaled=.95]{newtxtt}
%\usepackage[scaled=.8]{beramono}
%\renewcommand{\familydefault}{\sfdefault}
\KOMAoptions{fontsize=12pt}

%Espaciado de renglones: 1 (Espaciado simple). 
%\usepackage[singlespacing]{setspace} 
% KOMA lo tiene por defecto


%Espaciado de párrafos: un renglón libre.Cada punto aparte genera un párrafo. //Entonces no habrá sangrías.
\KOMAoptions{parskip=full}

%Ecuaciones  centradas.  Deben  aparecer  después  de  ser  citadas  en  el  texto. Numeración entre paréntesis justificada a la derecha. 

%Todas  las  figuras  deben  estar  centradas,  tener  un  Nº  de  figura  y  leyenda.  Deben aparecer después de ser citadas en el texto. 

%Encabezado de página: en todas las páginas exceptuando los inicios de capítulo. El encabezado debe incluir el número y título de capítulo en cursiva tamaño 9 y estar justificado a la izquierda.
\usepackage[automark,headsepline,draft=false]{scrlayer-scrpage}
\clearpairofpagestyles
\lohead{\headmark}
\setkomafont{pageheadfoot}{\itshape\footnotesize} % scriptsize es 8 si base es 12. Else: footnotesize, que es 10.
\pagestyle{scrheadings}

%Numeración  de  páginas  en  el  pié  de  página,  justificada  a  la  derecha,  número  no cursivo, tamaño 12. 
\ofoot[\pagemark]{\pagemark} % opcion incluye las planas
%\ofoot[]{\pagemark}
\setkomafont{pagenumber}{\upshape\normalsize} %12

%Tamaño de títulos:  
%//Quedan muy chicos. Deberíamos usar los sugeridos por defecto.
%De capítulo    16 en negrita 
\setkomafont{chapter}{\bfseries\Large} % 17,28pt

%De secciones  14 en negrita 
\setkomafont{section}{\bfseries\large} % 14,4pt

%Secundarios   14 en negrita 
\setkomafont{subsection}{\bfseries\large} % 14,4pt


%No referenciados  12 en negrita
%\setkomafont{subsubsection}{\itshape\bfseries\normalsize} % 12. Italics para más contraste con el texto común.
\setkomafont{subsubsection}{\bfseries\normalsize} % 12. Les puse color, así que no hace falta que estén en cursiva.

\setkomafont{minisec}{\itshape\bfseries\normalsize}

%Máxima cantidad de niveles de títulos referenciados: 3 
%0. Capítulo / Título
%1. Sección
%2. Subsección
%3. Subsubsección
\setcounter{tocdepth}{2}
\setcounter{secnumdepth}{\subsectionnumdepth}

%Después de la tapa celeste con ventana provista por la secretaría del Departamento de  Electrónica  y  Automática,  sigue  una  hoja  impresa  provista  por  la Comisión  de Trabajos Finales donde, en el espacio correspondiente a la ventana debe escribirse el título de trabajo final, el nombre de los autores y el año. 
 
%A continuación sigue una hoja donde se escribe: Universidad Nacional de San Juan,Facultad  de  Ingeniería,  Departamento  de  Electrónica  y  Automática,  el  título  del trabajo final, el nombre de los autores, el nombre de los asesores y el año. Luego sigue una hoja con los agradecimientos. En otra hoja se inicia la escritura del índice. Finalmente sigue el resto del trabajo. 

