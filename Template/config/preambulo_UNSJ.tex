% Clase del documento. KOMA-Script Report de base.

% Asumimos una impresión a dos caras, dejando un margen para la
% encuadernación y no respeteando los márgenes solicitados.
% Las opciones 'draft' o 'final' permiten compilar un documento de
% borrador o el final, respectivamente. En el borrador, para menor
% tamaño y mayor velocidad de compilación, no se incluyen las imágenes,
% entre otras cosas. Comentar y descomentar a gusto.

%\documentclass[draft,twoside=yes,bibliography=totoc]{scrreprt}
\documentclass[final,twoside=yes,bibliography=totoc]{scrreprt}

% Otras notas: 
%   Usar 'spanish' como opción global del documento provoca errores en los
%   paquetes 'babel' y 'csquotes'.
% 
%   'bibliography=totoc' debe estar como opción global por requerimiento de
%   BibLaTeX. No funciona como \KOMAoptions{bibliography=totoc} o
%   \KOMAoption{bibliography}{totoc}. Me parece que es culpa de Biblatex o
%   Biber y no de Koma.


% Las numeraciones de secciones y figuras no tienen un punto al final
% (nosotros usamos '2.1' en lugar de '2.1.'). Necesita también
% 'es-nosectiondot' o 'es-nolayout' en babel.
\KOMAoptions{numbers=noendperiod} 


%\KOMAoptions{headinclude=true}
%\KOMAoptions{footinclude=true}
%\KOMAoptions{DIV=15}


% Elegimos que los capítulos empiecen en páginas izquierdas o derechas 
% indistitamente, para ahorrar papel. Otras opciones: 'left' y 'right'.
\KOMAoptions{open=any}

 
% El commando \raggedbottom puede estar en cualquier parte del documento y sirve
% para que todas las páginas tengan una longitud vertical natural de su contenido, en
% lugar de estirar algunos espacios para intentar que frente y dorso ocupen más o 
% menos el mismo espacio. Esta opción está por defecto en documentos de un solo lado,
% pero no en los de dos lados
\raggedbottom



% IDIOMA
% Por problemas que no recuerdo, usamos PdfLaTeX como motor y Babel para
% el idioma. En caso contrario, Polyglossia.
% Polyglossia - An alternative to babel for XeLaTeX and LuaLaTeX
% https://ctan.org/pkg/polyglossia
% Este paquete es para mejor soporte de idiomas.
%\usepackage[]{polyglossia}
%\setdefaultlanguage[]{spanish}
%\setotherlanguage[variant=american]{english}
% Para escribir en inglés, usar el comando \textenglish{texto}
% o el entorno english: \begin{english}[opciones]{}\end{english}


% El paquete 'inputenc' nos permitirá usar UTF8 como codificación de entrada.
% Esto nos permite escribir nativamente en UTF8 sin preocuparnos.
% Importante: las fuentes cargadas deben soportar los caracteres usados.
\usepackage[utf8]{inputenc}



% El paquete babel carga las reglas tipográficas correspondientes a cada idioma
% que usemos en el documento. 'es-nosectiondot' es para no agregar puntos finales
% en los números de sección y figura. 'es-nolists' es para no modificar los
% estilos de lista de Koma.
% Puede ser interesante lo de la opción 'mexico', pero no está funcionando...
\usepackage[english,spanish,es-nosectiondot,es-nolists]{babel}
%\usepackage[english,spanish,mexico-com,es-nosectiondot,es-nolists]{babel}
%\usepackage[english,spanish,es-nosectiondot,es-nolayout]{babel}

% Seleccionamos el idioma por defecto de ahora en adelante.
\selectlanguage{spanish}

% Un atajo:
\babeltags{en = english}
% Y entonces \texten{texto} servirá para inglés. igual usaremos macro \ingles{texto}.
% Para cosas largas: \begin{en} y \end{en}


% Para usar 'tablas' en lugar de 'cuadros'.
% Alternativa si usamos Koma-Script:
\renewcaptionname{spanish}{\listtablename}{Índice de tablas}
\renewcaptionname{spanish}{\tablename}{Tabla}

% Alternativa propuesta por Polyglossia:
%\usepackage{etoolbox}
%\gappto\captionsspanish{\renewcommand{\tablename}{Tabla}}
%\gappto\captionsspanish{\renewcommand{\listtablename}{Índice de tablas}}

% En caso de Babel:
%\addto\captionsspanish{
%	\def\tablename{Tabla}
%	\def\listtablename{\'Indice de tablas}
%   \def\figurename{Figura}
%}



% FORMATO
% Con * los requerimientos de la Comisión.

% * Tamaño de papel: A4 
% Esto está por defecto en KOMA-Script.


% * Márgenes: Superior = Inferior = 2,5cm; Izquierdo = 2cm; Derecho = 2cm 
% Lo desplazo medio centímetro hacia afuera, para agregar espacio para
% la encuadernación.
\usepackage[inner=2.5cm,top=2.5cm,outer=1.5cm,bottom=2.5cm,bindingoffset=0cm,footskip=1.5cm]{geometry}


% * Impresión en doble faz.
% Está arriba con twoside=yes


% * Letra tamaño 12. Tipo Arial o Times New Roman. 
\KOMAoptions{fontsize=12pt}

% Arial y Times New Roman son propietarias y no necesariamente los alumnos las 
% tienen. Pero existen muchos clones o equivalentes.

%\usepackage{helvet}
%\usepackage[scaled]{helvet} 
%Helvetica is actually somewhat larger than other typefaces of the same nominal size.  As a result, mixing, e.g., Times and Helvetica within running text may look bad. This  can  be  fixed  by  loading  the  package  with  the  option [scaled=〈scale〉], for instance: \usepackage[scaled=.92]{helvet}.  As a result, the font family phv (Helvetica) will be scaled down to 92% of its ‘natural’ size, which is suitable for use with Adobe Times.  Specifying [scaled] alone is equivalent to [scaled=0.95].

% No recuerdo la causa pero
% Elegimos la familia NewTx para texto, matemática y texto monoespaciado.
\usepackage{newtxtext, newtxmath}
% Fuente monoespaciada ligeramente escaladada para tener el mismo tamaño que el resto.1
\usepackage[zerostyle=d,scaled=.95]{newtxtt}
%\usepackage[scaled=.8]{beramono}

% Por defecto, el cuerpo del texto está en la familia con serifa (tipo Times), mientras
% que la sin serifa (paloseco) se usa en títulos y detalles. Podría cambiarse con el siguiente comando, pero habría que cambiar varias cosas más. Lo recomendable es que el cuerpo esté en una fuente con serifa.
%\renewcommand{\familydefault}{\sfdefault}



% * Espaciado de renglones: 1 (Espaciado simple). 
%\usepackage[singlespacing]{setspace} 
% Koma lo tiene por defecto


%* Espaciado de párrafos: un renglón libre.Cada punto aparte genera un párrafo. //Entonces no habrá sangrías. O usás sangrado o usás espaciado.
\KOMAoptions{parskip=full}


% * Ecuaciones  centradas.  Deben  aparecer  después  de  ser  citadas  en  el  texto. Numeración entre paréntesis justificada a la derecha. 
% Por defecto en LaTeX.


% * Todas  las  figuras  deben  estar  centradas,  tener  un  Nº  de  figura  y  leyenda.  Deben aparecer después de ser citadas en el texto.
% Ver ejemplos en los capítulos y comentarios en secciones/macros.tex.


% * Encabezado de página: en todas las páginas exceptuando los inicios de capítulo. El encabezado debe incluir el número y título de capítulo en cursiva tamaño 9 y estar justificado a la izquierda.
% 'headsepline' es la línea que separa la cabecera del cuerpo.
\usepackage[automark,headsepline,draft=false]{scrlayer-scrpage}
\clearpairofpagestyles

% Justificado a la izquierda e ambos lados:
\lohead{\headmark} % left even
\lehead{\headmark} % left odd
%\ohead{\headmark} % outer

% scriptsize es 8 si el tamaño base de la fuente es 12.
% footnotesize es 10 en el mismo caso.
\setkomafont{pageheadfoot}{\itshape\footnotesize} 
\pagestyle{scrheadings}


%* Numeración  de  páginas  en  el  pié  de  página,  justificada  a  la  derecha,  número  no cursivo, tamaño 12. 
\setkomafont{pagenumber}{\upshape\normalsize} %12

% Elegimos que estén del lado externo de las páginas.
\ofoot[\pagemark]{\pagemark} % opcion incluye las páginas con estilo plano.
%\ofoot[]{\pagemark} % no incluye las páginas con estilo plano (aperturas de capítulo).



% Tamaño de títulos. Con base 12pt, la opción 'small' de 'headings
% queda en (ver capítulo 3 de Koma-Script):
% Capítulo: \Large 17.28
% Sección: \large 14.4
% Subsección: \normalsize 12
% Subsubsección: \normalsize 12
% Minisec: \normalsize 12
% Par: \normalsize 12
%\KOMAoptions{headings}


% * Tamaño de títulos:  
%//Quedan muy chicos. Deberíamos usar los sugeridos por defecto en KOMA-Script.

% * De capítulo    16 en negrita 
\setkomafont{chapter}{\bfseries\Large} % 17,28pt

% * De secciones  14 en negrita 
\setkomafont{section}{\bfseries\large} % 14,4pt

% * Secundarios   14 en negrita 
\setkomafont{subsection}{\bfseries\large} % 14,4pt

% * No referenciados  12 en negrita
%\setkomafont{subsubsection}{\itshape\bfseries\normalsize} % 12. Italics para más contraste con el texto común.
\setkomafont{subsubsection}{\bfseries\normalsize} % 12. Les puse color, así que no hace falta que estén en cursiva. Ver en config/preambulo_otros.tex. 

% minisec es un encabezado provisto por KOMA-Script. Es más cercano al texto, de la menor importancia posible, pero mayor que \paragraph{}. Lo usamos con un color distinto a los títulos, porque tiene el mismo tamaño que el texto (\normalsize). Ver en config/preambulo_otros.tex. 
\setkomafont{minisec}{\normalsize}

% Ver 'headings=selection' en ayuda de Koma-Script, capítulo 3. Podría ser bueno usar
% 'headings = small'.



% * Máxima cantidad de niveles de títulos referenciados: 3 (hasta subsección)
%0. Capítulo / Título
%1. Sección
%2. Subsección
%3. Subsubsección
\setcounter{tocdepth}{2}
\setcounter{secnumdepth}{\subsectionnumdepth}

% * Después de la tapa celeste con ventana provista por la secretaría del Departamento de  Electrónica  y  Automática,  sigue  una  hoja  impresa  provista  por  la Comisión  de Trabajos Finales donde, en el espacio correspondiente a la ventana debe escribirse el título de trabajo final, el nombre de los autores y el año. 

% * A continuación sigue una hoja donde se escribe: Universidad Nacional de San Juan,Facultad  de  Ingeniería,  Departamento  de  Electrónica  y  Automática,  el  título  del trabajo final, el nombre de los autores, el nombre de los asesores y el año. Luego sigue una hoja con los agradecimientos. En otra hoja se inicia la escritura del índice. Finalmente sigue el resto del trabajo. 
