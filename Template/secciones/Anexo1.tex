\chapter{Organización de los archivos}\label{anexo1}

\begin{labeling}[—]{\texttt{/tfinal\_AGUADO.pdf}}
\setkomafont{labelinglabel}{\ttfamily}
\item [/inicializar.m] Archivo de inicialización (ver anexo \ref{anexo2}).
\item [/tfinal\_AGUADO.pdf] Este documento.
\item [/clases/] Archivos de clase de preprocesadores, segmentadores, clasificadores, clases principales, enumeraciones y otros.
\item [/experimentos/] Archivos de cada experimento realizado. Cada carpeta contiene los archivos de configuración del experimento y planillas con los resultados generados.
\item [/funciones/] Funciones principales y auxiliares desarrolladas en \nombre{Matlab}.
\item [/GUI/] Interfaz gráfica de usuario.
\item [/imagenes/] Los conjuntos de imágenes generados y sus metadatos. \nombre{set1} sirvió para experimentación inicial, con \nombre{set2} se desarrollaron los primeros algoritmos y \nombre{set3} es el conjunto principal con el que se evaluó el sistema (ver sección \ref{captura:conjuntodeimagenes}). De este último se incluyen versiones preprocesadas ---recortadas, sin distorsión y con máscaras binarias---. Los archivos \texttt{.csv} son las listas completas de imágenes y de los conjuntos de entrenamiento y evaluación; la planilla \texttt{.xlsx} tiene información sobre la clasificación manual.\todo{NO OLVIDAR ADJUNTAR}
\item [/info/] Documentos e imágenes que fueron útiles para la realización de este trabajo.\todo{NO OLVIDAR ADJUNTAR}
\item [/otros/] Archivos correspondientes a programas de terceros requeridos o utilizados: \nombre{Balu}, \nombre{YAMLMatlab} y \nombre{multic}.
\item [/scripts/] Programas auxiliares usados durante el trabajo. No son necesarios para la ejecución de experimentos; sirvan para referencia futura.
\end{labeling}