% Este comando define el tipo de página como vacía, lo que implica que no habrán
% números de página ni cabeceras ni pies de página. Solo aplica a una página. En
% caso de ser más, usar \pagestyle{option}.
% Las opciones en LaTeX son:
%   empty           vacía
%   plain             sólo números de página
%   headings      números de página y cabecera
%   myheadings   personalizada (con el paquete fancyhdr, por ejemplo)
%
% Para esta plantilla con Koma-Script usar
%   empty                   vacía
% plain.scrheadings  sólo números de página
%   scrheadings         números de página y cabecera
%\thispagestyle{empty} % Ya está en tfinal.tex.
      
{
    \itshape
    \small
    \textbf{Agradezco a todos, por todo.}
    
    Dicho eso, expando mis agradecimientos a quienes de alguna u otra manera contribuyeron a este Trabajo Final y a todos los años de estudio que le preceden.
    %\vfill % espacio vertical elástico
    \vspace{1cm}
    \begin{flushright}
        El autor
    \end{flushright}
}

