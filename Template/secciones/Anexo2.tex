\chapter{Uso de los programas}\label{anexo2}

Los programas requieren que se definan las variables \texttt{carpeta\_base} y \texttt{carpeta\_imagenes} con las rutas \textbf{absolutas} a la carpeta raíz de los programas (raíz de las carpetas \texttt{GUI}, \texttt{clases}, \textellipsis) y a la carpeta raíz de las imágenes respectivamente (raíz de las carpetas \texttt{set1}, \texttt{set2}, \texttt{set3}, \textellipsis).


Para ello, como requisito inicial, se debe editar el archivo \texttt{inicializar.m}, disponible en el directorio raíz.

Si bien se pueden editar y ejecutar los experimentos disponibles en la carpeta \texttt{/experimentos}, recomiendo ejecutar la interfaz gráfica de usuario para poder apreciar lo que están haciendo los programas. Para ello:

\begin{enumerate}
\item Editar y ejecutar \texttt{inicializar.m}.
\item Editar \texttt{/GUI/gui.m} y seleccionar (comentando y descomentando líneas) el conjunto de imágenes a utilizar y las etapas del algoritmo.
\item Ejecutar \texttt{gui.m}.
\end{enumerate}
